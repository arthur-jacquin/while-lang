%%% Ce document n'est pas à jour.

\documentclass{beamer}
\title{Un interpréteur pour le langage WHILE}
\author{Arthur Jacquin}
\date{}

\usepackage{xcolor}
\definecolor{beaverred}{RGB}{204,0,0}
\definecolor{beaverblue}{RGB}{0,68,204}
\definecolor{beavergreen}{RGB}{0,153,0}

\usepackage{listings}
\lstdefinestyle{simple}{
    extendedchars=true,
    basicstyle=\ttfamily,
    backgroundcolor=,
    commentstyle=\color{beavergreen},
    keywordstyle=\color{beaverblue},
    numberstyle=\ttfamily\color[rgb]{0.5,0.5,0.5},
    breakatwhitespace=false,
    breaklines=true,
    captionpos=b,
    keepspaces=true,
    numbers=left,
    numbersep=7pt,
    showspaces=false,
    showstringspaces=false,
    showtabs=false,
    tabsize=4,
    columns=fixed
}
\lstset{style=simple}

\usetheme{default}
\usecolortheme{beaver}
\setbeamertemplate{navigation symbols}{}
\setbeamertemplate{footline}[frame number]
\setbeamertemplate{itemize items}{\color{beaverred}$\bullet$}

\begin{document}

\frame{\titlepage}

\begin{frame}
\frametitle{Sommaire}
\begin{itemize}
\item Introduction
    \begin{itemize}
    \item Objet de la présentation
    \item Fonctionnement général d'un interpréteur
    \item Langage retenu : OCaml
    \end{itemize}
\item Conception
    \begin{itemize}
    \item Définir la nature des objets
    \item Implémenter la logique interprétative
    \item Concevoir simultanément syntaxe et parsing
    \end{itemize}
\item Parsing
    \begin{itemize}
    \item RPN pour les expressions arithmétiques
    \item Fonctions auxiliaires de recherche de motif
    \item Traitement global
    \end{itemize}
\item Exemples
    \begin{itemize}
    \item \texttt{factorial.while}
    \item \texttt{pgcd.while}
    \end{itemize}
\item Pour aller plus loin
\end{itemize}
\end{frame}

\begin{frame}
\frametitle{Introduction - Objet de la présentation}
\begin{itemize}
\item Point de départ : langage WHILE tel qu'étudié lundi
\item Objectif : concevoir et mettre en oeuvre une implémentation
\item Exemples de scripts à la fin
\end{itemize}
\end{frame}
% Le but c'est d'expliquer un peu ce que j'ai fait
% Il y a pas mal de matière donc je vais essayer de couvrir les points les plus intéressants
% Si vous avez des questions n'hésitez pas à m'interrompre, vraiment

\begin{frame}
\frametitle{Introduction - Fonctionnement général d'un interpréteur}
\begin{itemize}
\item Code source
\item PARSING $\longrightarrow$ Objects informatiques
\item INTERPRÉTATION $\longrightarrow$ Résultats
\end{itemize}
\end{frame}

\begin{frame}
\frametitle{Introduction - Langage retenu : OCaml}
\begin{itemize}
\item Typage fort, possibilité de définir de nouveaux types
\item Très bon {\it pattern matching}, même sur les types non standards
\item S'interprète ou se compile, au choix
\end{itemize}
\end{frame}

\begin{frame}[fragile]
\frametitle{Conception - Définir la nature des objets}
\begin{itemize}
\item Très important, typage explicite à priviliégier par la suite
\item Choix de l'ensemble des identifieurs
\end{itemize}
\begin{scriptsize}
\lstinputlisting[language=ML, firstline=13, firstnumber=13, lastline=24]{interpreter.ml}
\end{scriptsize}
\end{frame}
\begin{frame}[fragile]
\frametitle{Conception - Définir la nature des objets}
\begin{scriptsize}
\lstinputlisting[language=ML, firstline=26, firstnumber=26, lastline=48]{interpreter.ml}
\end{scriptsize}
\end{frame}
\begin{frame}[fragile]
\frametitle{Conception - Définir la nature des objets}
\begin{itemize}
\item Fonction principale
\end{itemize}
\begin{scriptsize}
\lstinputlisting[language=ML, firstline=243, firstnumber=243, lastline=249]{interpreter.ml}
\end{scriptsize}
% Appel en ligne de commande
\end{frame}

\begin{frame}[fragile]
\frametitle{Conception - Implémenter la logique interprétative}
\begin{scriptsize}
\lstinputlisting[language=ML, firstline=54, firstnumber=54, lastline=70]{interpreter.ml}
\end{scriptsize}
\end{frame}
% On peut remarquer que l'ordre d'implémentation compte, puisque le typage des objets n'est pas indépendant

% Jusque là rien de très surprenant sur 1. et 2., c'est exactement le travail qu'on a fait lundi

\begin{frame}
\frametitle{Conception - Concevoir simultanément syntaxe et parsing}
\begin{itemize}
\item Simple à utiliser ET simple à analyser (efficacité)
\item Prendre en compte les modalités de l'interface avec le fichier
\end{itemize}
\end{frame}
% Décrire ce qui va être abordé

\begin{frame}
\frametitle{Parsing - RPN pour les expressions arithmétiques}
\begin{itemize}
\item Les expressions arithmétiques peuvent être complexes
\item Élément très courant : ce serait bien de trouver une méthode efficace
\item Une solution : la notation polonaise inversée (RPN) lève toute ambiguité
% la notation suffixe permet de reconstruire l'arbre associé à l'expression
% (étant donné un parcours suffixe/préfixe d'arbre où l'on sait distinguer les
% noeuds internes des feuilles, comme ici avec les litéraux (variable, entier)
% et les opérateurs (+, *, -), il y a unicité de l'arbre l'ayant généré; c'est
% pas le cas avec un parcours infixe, dont c'est un peu stupide d'utiliser cette
% convention en maths parce que ça génère des ambiguités, après il faut parentheser
% pour se comprendre)
\item Permet une lecture linéaire de l'expression, avec accumulation des litéraux (variables, entiers) dans une pile et réduction par les opérateurs (+, -, *)
% Tout ça c'est très bien, on est passé en temps linéaire, mais faire
% une implémentation efficace reste pas simple, vous regarderez plus en déatil si vous voulez
\item Mon implémentation ne fonctionne pas avec les nombres négatifs
\end{itemize}
\end{frame}

\begin{frame}
\frametitle{Parsing - Fonctions auxiliaires de recherche de motif}
% Avec les booléens la RPN ça pourrait encore se faire, de toute façon c'est
% possible avec tout, même les structures de contrôle comme while, mais
% c'est moins naturel, on veut quand même que le script ressemble à une suite
% d'instruction
\begin{itemize}
\item Moteur de recherche d'expression régulière : échoue à extraire les "arguments"
\end{itemize}
\begin{scriptsize}
\lstinputlisting[language=ML, firstline=199, firstnumber=199, lastline=212]{interpreter.ml}
\end{scriptsize}
% Commenter les signatures, les recherches successives
\end{frame}


\begin{frame}
\frametitle{Parsing - Traitement global}
% Jusqu'à maintenant, avec les objets à parser, on se doutait bien que
% c'était une fonction plus grosse qui allait délimiter l'objet puis le
% donner à analyser aux fonctions déjà faites
% Maintenant qu'on s'approche des plus gros objets (eg les structures
% de controle), il faut se demander comment on peut accéder au texte
% même d'un autre fichier.
\begin{itemize}
\item Accès au code source ? Utilisation d'un \texttt{input channel} qui permet de récupérer les lignes les unes après les autres
% ocaml ne propose pas grand chose, nous laisse directement interagir avec le système
% primitive fournie par ocaml, qui ne fait que nous la transmettre du kernel: input channel
% quelque chose où on peut lire dedans, mais le canal se "consumme", cad qu'on
% avance dans le fichier sans pouvoir revenir en arrière
% en fait c'est un pointeur, quand on lui demande de lire, il lit jusqu'au prochain
% saut de ligne, nous communique ce qu'il a lu, et saute à la ligne suivante
% S'il détecte que c'est la fin de fichier, il nous envoie une erreur
\item Problème d'expressions sur plusieurs lignes : récursivité
% Pour parser les instructions on aimerait bien faire la même chose que
% pour les expressions booléennes, mais il y a un problème pour while et
% if-then-else: l'instruction occupe plusieurs lignes, alors qu'on les récupère
% une par une
% la solution c'est de faire une fonction récursive, qui prend en entrée un
% canal d'entrée, traite un bloc de code, et lorsque la fin de ce bloc est détectée
% renvoie la commande associée
\end{itemize}
\end{frame}
\begin{frame}[fragile]
\frametitle{Parsing - Traitement global}
\begin{scriptsize}
\lstinputlisting[language=ML, firstline=214, firstnumber=214, lastline=233]{interpreter.ml}
\end{scriptsize}
% travail préliminaire de
%   gestion des erreurs
%   détection de fin de bloc
%   ignorer commentaires et lignes vides
% une fois qu'on a isolé une instruction simple on revient aux méchanismes précédents
% concaténation des instructions
% fermer le canal d'entrée (très important)
\end{frame}

\begin{frame}[fragile]
\frametitle{Exemples - \texttt{factorial.while}}
\begin{scriptsize}
\lstinputlisting[firstline=1, firstnumber=1, lastline=19]{factorial.while}
\end{scriptsize}
\end{frame}

\begin{frame}[fragile]
\frametitle{Exemples - \texttt{pgcd.while}}
\begin{scriptsize}
\lstinputlisting[firstline=1, firstnumber=1, lastline=21]{pgcd.while}
\end{scriptsize}
\end{frame}

\begin{frame}
\frametitle{Pour aller plus loin}
% la grosse limitation c'est l'absence totale de routines, que ce
% soit sous la forme de fonction, procédures, sauts dans le code...
\begin{itemize}
\item \url{https://github.com/arthur-jacquin/while-lang}
\item Utiliser une fonction récursive dans \texttt{parse\_arithm}
\item Corriger une erreur dans le traitement des expressions booléennes
\end{itemize}
\end{frame}

\end{document}
